\documentclass{article}[12pt]
\usepackage{fontspec}   %加這個就可以設定字體
\usepackage{xeCJK}       %讓中英文字體分開設置
\usepackage{indentfirst}
\usepackage{listings}
\usepackage[newfloat]{minted}
\usepackage{float}
\usepackage{graphicx}
\usepackage{caption}
\usepackage{fancyhdr}
\usepackage{hyperref}
\usepackage{amsmath}
\usepackage{multirow}
\usepackage[dvipsnames]{xcolor}
\usepackage{graphicx}
\usepackage{tabularx}
\usepackage{booktabs}
\usepackage{caption}
\usepackage{subcaption}
\usepackage{pifont}
\usepackage{amssymb}
\usepackage{titling}

\usepackage{pdftexcmds}
\usepackage{catchfile}
\usepackage{ifluatex}
\usepackage{ifplatform}

\usepackage[breakable, listings, skins, minted]{tcolorbox}
\usepackage{etoolbox}
\setminted{fontsize=\footnotesize}
\renewtcblisting{minted}{%
    listing engine=minted,
    minted language=cpp,
    listing only,
    breakable,
    enhanced,
    minted options = {
        linenos, 
        breaklines=true, 
        breakbefore=., 
        % fontsize=\footnotesize, 
        numbersep=2mm
    },
    overlay={%
        \begin{tcbclipinterior}
            \fill[gray!25] (frame.south west) rectangle ([xshift=4mm]frame.north west);
        \end{tcbclipinterior}
    }   
}

\usepackage[
top=1.5cm,
bottom=1.5cm,
left=2.5cm,
right=2.5cm,
includehead,includefoot,
heightrounded, % to avoid spurious underfull messages
]{geometry} 

\usepackage[autostyle]{csquotes}

\usepackage[
    backend=biber,
    style=ieee,
    natbib=true,
    doi=true,
    eprint=false
]{biblatex}
\addbibresource{project3.bib}

\newenvironment{code}{\captionsetup{type=listing}}{}
\SetupFloatingEnvironment{listing}{name=Code}
\usepackage[moderate]{savetrees}


\title{TCG Project 3: NoGo}
\author{李杰穎 110550088}
\date{\today}


\setCJKmainfont{Noto Serif TC}

\iflinux
\setmonofont[Mapping=tex-text]{Cascadia Code}
\fi

\ifwindows
\setmonofont[Mapping=tex-text]{Consolas}
\fi

\XeTeXlinebreaklocale "zh"             %這兩行一定要加,中文才能自動換行
\XeTeXlinebreakskip = 0pt plus 1pt     %這兩行一定要加,中文才能自動換行

\setlength{\parindent}{2em}
\setlength{\parskip}{2em}
\renewcommand{\baselinestretch}{1.25}
\setlength{\droptitle}{-10em}   % This is your set screw

\begin{document}

\maketitle



\section{MCTS}

Monte-Carlo Tree Search (MCTS) 為一種搜尋演算法,透過 UCT 公式 (\autoref{eq: uct}) 進行 Selection,再從 selection 到的節點進行 expansion 和 simulation,最後再透過 simulation 的結果 backpropagation 到路徑上各個節點。以上的步驟稱為一個 iteration,我們可以透過改變 iteration 的次數,控制 MCTS 執行的時間,此種特性使 MCTS 在許多有限制總思考時間的遊戲如 Go 和本次作業的 NoGo 成為一個非常適合的搜尋演算法。

值得注意的是,MCTS 最終在挑選最佳的下一步時,是挑選探訪過最多次的節點。

\begin{figure}[H]
\centering
\includegraphics[width=0.7\linewidth]{"img/mcts"}
\caption{Monte-Carlo Tree Search}
\label{fig:mcts}
\end{figure}

\begin{equation}
\frac{w_i}{n_i}+c\,\sqrt{\frac{\ln N_i}{n_i}}
\label{eq: uct}
\end{equation}

在本次作業中,我將每步的計算時間設為 1.5 秒,且每步最多執行 13500 個 iteration,後者的設定是避免在終局的時候搜尋過久,可能會導致 TLE 的問題。而 UCT 的 $c$ 我則參考 \cite{She2013},將其設為 0.75。

在本章中,為了實驗的快速進行,我先將每步的計算時間設為 0.1 秒。而後使用 Parallel MCTS 時,則會設為前述所提到的 1.5 秒。

以下為實驗數據皆為雙方各執黑執白 10 場,總共 20 場比賽的對戰結果:

\begin{table}[H]
\centering
\caption{Non-parallel MCTS,每一步思考時間為 0.1 秒的實驗結果}
\label{tab:mcts-1}
\begin{tabular}{@{}ccccc@{}}
\toprule
       & 執黑 & 執白 & 總勝場 & 勝率     \\ \midrule
Random & 10 & 10 & 20  & 100 \% \\
Weak   & 8  & 7  & 15  & 75 \%  \\
Strong & 0  & 0  & 0   & 0 \%   \\ \bottomrule
\end{tabular}
\end{table}

\section{Parallel MCTS}

Parallel MCTS,我參考了 \cite{5654650},文中提到四種 parallelization 的技術中,root parallelization 的效果最好,且實作也簡單。故在編寫 parallel MCTS 的程式碼時,即採用 root parallelization 的技術。

Root parallelization 的概念大致如下:每個 thread 都會進行獨立的 MCTS,不受其他 thread 的影響,而在最後決定 best action 時,則將各個 thread 的結果合併,挑出各個樹加起來探訪最多次的節點。

Parallel MCTS 的實驗結果如下,thread 的個數為 4,每步的計算時間為 0.1 秒:


\begin{table}[H]
\centering
\caption{Parallel MCTS,每一步思考時間為 0.1 秒、thread 個數為 4 的實驗結果}
\label{tab:mcts-2}
\begin{tabular}{@{}ccccc@{}}
\toprule
       & 執黑 & 執白 & 總勝場 & 勝率     \\ \midrule
Random & 10 & 10 & 20  & 100 \% \\
Weak   & 9  & 10 & 19  & 95 \%  \\
Strong & 1  & 0  & 1   & 5\%    \\ \bottomrule
\end{tabular}
\end{table}

可以發現在相同執行時間的情況下,與 weak 和 strong player 對戰的勝率皆有提升。

最後,我將每步的計算時間設為 1.5 秒,而跟 strong player 對戰 20 場的結果如下:

\begin{table}[H]
\centering
\caption{Parallel MCTS,每一步思考時間為 1.5 秒、thread 個數為 4 的實驗結果}
\label{tab:mcts-3}
\begin{tabular}{@{}ccccc@{}}
\toprule
       & 執黑 & 執白 & 總勝場 & 勝率   \\ \midrule
Strong & 9  & 6  & 15  & 75\% \\ \bottomrule
\end{tabular}
\end{table}

可以發現勝率達到了 75 \%

\section{未來展望}

本次作業中,我並沒有嘗試 RAVE 這個技巧,在 \cite{She2013} 有提到綜合 MCTS 及 RAVE 的 AI 與只有 MCTS 的 AI 進行對戰的勝率高達 99.8 \%。或許可以多加嘗試。

另外,我也沒有將總思考時間完美的控制在 40 秒左右,這點也是未來可以改進的方向。

\printbibliography

\end{document}
