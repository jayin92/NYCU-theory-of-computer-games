\documentclass{article}[12pt]
\usepackage{fontspec}   %加這個就可以設定字體
\usepackage{xeCJK}       %讓中英文字體分開設置
\usepackage{indentfirst}
\usepackage{listings}
\usepackage[newfloat]{minted}
\usepackage{float}
\usepackage{graphicx}
\usepackage{caption}
\usepackage{fancyhdr}
\usepackage{hyperref}
\usepackage{amsmath}
\usepackage{multirow}
\usepackage[dvipsnames]{xcolor}
\usepackage{graphicx}
\usepackage{tabularx}
\usepackage{booktabs}
\usepackage{caption}
\usepackage{subcaption}
\usepackage{pifont}
\usepackage{amssymb}
\usepackage{titling}

\usepackage{pdftexcmds}
\usepackage{catchfile}
\usepackage{ifluatex}
\usepackage{ifplatform}

\usepackage[breakable, listings, skins, minted]{tcolorbox}
\usepackage{etoolbox}
\setminted{fontsize=\footnotesize}
\renewtcblisting{minted}{%
    listing engine=minted,
    minted language=cpp,
    listing only,
    breakable,
    enhanced,
    minted options = {
        linenos, 
        breaklines=true, 
        breakbefore=., 
        % fontsize=\footnotesize, 
        numbersep=2mm
    },
    overlay={%
        \begin{tcbclipinterior}
            \fill[gray!25] (frame.south west) rectangle ([xshift=4mm]frame.north west);
        \end{tcbclipinterior}
    }   
}

\usepackage[
top=1.5cm,
bottom=1.5cm,
left=2.5cm,
right=2.5cm,
includehead,includefoot,
heightrounded, % to avoid spurious underfull messages
]{geometry} 

\newenvironment{code}{\captionsetup{type=listing}}{}
\SetupFloatingEnvironment{listing}{name=Code}
\usepackage[moderate]{savetrees}


\title{TCG Project 1 Report}
\author{110550088 李杰穎}
\date{\today}


\setCJKmainfont{Noto Serif TC}

\iflinux
\setmonofont[Mapping=tex-text]{Cascadia Code}
\fi

\ifwindows
\setmonofont[Mapping=tex-text]{Consolas}
\fi

\XeTeXlinebreaklocale "zh"             %這兩行一定要加,中文才能自動換行
\XeTeXlinebreakskip = 0pt plus 1pt     %這兩行一定要加,中文才能自動換行

\setlength{\parindent}{0em}
\setlength{\parskip}{1.5em}
\renewcommand{\baselinestretch}{1.25}
\setlength{\droptitle}{-10em}   % This is your set screw

\begin{document}

\maketitle

\section{Method \& Implementation}

In the given template, we already have a random slider, which will choose action randomly, regardless of board state. This method gets 65.6 points. Obviously, this is not a good results.

A better method is to try all four actions, and see which action will get higher score. I called this greedy slider. The implementation of vanilla greedy slider is shown as \autoref{code: vanilla greedy slider}. This method gets 86.2 point, which is a huge improvement compared with random slider.

To further improve performance, I introduce a technique that I use when playing 2048, which is forbidden one of the directions, i.e. I never slide up when there is no need to do that. This will keep tiles on the bottom of board always be larger than those on top, making game harder to end. The implementation is shown as \autoref{code: forbid_greedy_slider}. After introducing this technique, the performance get better. This method gets 93.8 point.

Noteworthy, I discover that the forbidden direction actually influence the final score given the judge. After some experiments, I found that the highest score I can get is 94.7 points, when forbidden sliding down.

Finally, I further increase the number of forbidden actions to two, after some experiments, I found that the highest score I can get is 95 points, when forbidden slides up and left, in other word, the agent only slides down and right if those actions is valid. The implementation of this method is shown as \autoref{code: greedy_slider}.

\begin{code}
\captionof{listing}{\texttt{class vanilla\_greedy\_slider} in \texttt{agent.h}}
\label{code: vanilla greedy slider}
\begin{minted}
class vanilla_greedy_slider : public random_agent {
public:
    vanilla_greedy_slider(const std::string& args = "") : random_agent("name=slide role=slider " + args),
        opcode({ 0, 1, 2, 3 }) {}

    virtual action take_action(const board& before) {
        board::reward max_reward = -1;
        int max_op;
        for (int op : opcode) {
            board::reward reward = board(before).slide(op);
            if(reward >= max_reward){
                max_reward = reward;
                max_op = op;
            }
        }
        if(max_reward == -1){
            return action();
        } else {
            return action::slide(max_op);
        }
    }

private:
    std::array<int, 4> opcode;
};
\end{minted}
\end{code}

\begin{code}
\captionof{listing}{\texttt{class forbid\_greedy\_slider} in \texttt{agent.h}}
\label{code: forbid_greedy_slider}
\begin{minted}
class forbid_greedy_slider : public random_agent {
public:
    forbid_greedy_slider(const std::string& args = "") : random_agent("name=slide role=slider " + args),
        opcode({ 1, 2, 3 }), alter_opcode({ 0 }) {}

    virtual action take_action(const board& before) {
        board::reward max_reward = -1;
        int max_op;
        for (int op : opcode) {
            board::reward reward = board(before).slide(op);
            if(reward >= max_reward){
                max_reward = reward;
                max_op = op;
            }
        }
        if(max_reward == -1){
            max_reward = -1;
            for (int op : alter_opcode) {
                board::reward reward = board(before).slide(op);
                if(reward >= max_reward){
                    max_reward = reward;
                    max_op = op;
                }
            }
            if(max_reward != -1){
                return action::slide(max_op);
            }
            return action();
        } else {
            return action::slide(max_op);
        }
    }

private:
    std::array<int, 3> opcode;
    std::array<int, 1> alter_opcode;
};
\end{minted}
\end{code}

\begin{code}
\captionof{listing}{\texttt{class greedy\_slider} in \texttt{agent.h}}
\label{code: greedy_slider}
\begin{minted}
class greedy_slider : public random_agent {
public:
    greedy_slider(const std::string& args = "") : random_agent("name=slide role=slider " + args),
        opcode({ 1, 2 }), alter_opcode({3, 0}) {}

    virtual action take_action(const board& before) {
        board::reward max_reward = -1;
        int max_op;
        for (int op : opcode) {
            board::reward reward = board(before).slide(op);
            if(reward >= max_reward){
                max_reward = reward;
                max_op = op;
            }
        }
        if(max_reward == -1){
            max_reward = -1;
            for (int op : alter_opcode) {
                board::reward reward = board(before).slide(op);
                if(reward >= max_reward){
                    max_reward = reward;
                    max_op = op;
                }
            }
            if(max_reward != -1){
                return action::slide(max_op);
            }
            return action();
        } else {
            return action::slide(max_op);
        }
    }

private:
    std::array<int, 2> opcode;
    std::array<int, 2> alter_opcode;
};
\end{minted}
\end{code}

\section{Conclusion}

After experiments, we found that the method that can get highest points is to forbid two of the actions. This will make larger tiles stay in one of the four corners.



\end{document}
